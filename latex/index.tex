\documentclass[a4paper, 12pt]{report}


\usepackage[czech]{babel} % czech language
\usepackage{amssymb} % for math symbols
\usepackage{amsmath} % for math symbols
\usepackage{pdfpages} % for including pdf files
\usepackage{microtype} % better text rendering
\usepackage[T1]{fontenc} % better text rendering
\usepackage{graphicx} % for images
\usepackage[hidelinks]{hyperref} % for clickable links, hidelinks hides the ugly boxes
\usepackage[a4paper,width=160mm,top=25mm,bottom=25mm,bindingoffset=6mm]{geometry} % page layout
\usepackage[pagestyles]{titlesec} % for customizing chapter titles
\usepackage{parskip} % for no indent and space between paragraphs
\usepackage{enumitem} % for customizing lists
\usepackage{fancyhdr} % for customizing headers and footers
\usepackage{bm} % for bold math symbols
\usepackage{tocloft} % for customizing table of contents
\usepackage{multicol} % for multiple columns
\usepackage{hyperref} % for clickable links
\usepackage{textcase} % for uppercasing text
\usepackage{tikz} % for drawing
\usepackage{float} % for floating images

% toc font
\renewcommand{\cfttoctitlefont}{\normalfont\Huge\sffamily\bfseries}
\renewcommand{\cftchapfont}{\sffamily\normalsize}
\renewcommand{\cftsecfont}{\sffamily\normalsize}



% for customizing bibliography title
\addto{\captionsczech}{\renewcommand{\bibname}{Seznam zdrojů}}

% for customizing chapter titles
\titleformat{\chapter}[display]{\normalfont\sffamily\bfseries}{}{0pt}{\Huge}
\newpagestyle{mystyle}
{\sethead[\thepage][][\chaptertitle]{}{}{\thepage}}
\pagestyle{mystyle}

\fancypagestyle{plain}{
    \renewcommand{\headrulewidth}{0pt}
    \fancyhf{}
    \fancyfoot[C]{\thepage}
}

\pagestyle{fancy}
\fancyhead{}
\fancyhead[R]{\rightmark}
\fancyhead[L]{\thechapter\ \chaptername}

\renewcommand{\sectionmark}[1]{\markright{#1}}

% for customizing section titles
\titleformat{\section}{\normalfont\Large\sffamily\bfseries}{\thesection}{1em}{}

% for customizing subsection titles
\titleformat{\subsection}{\large\sffamily\bfseries}{\thesubsection}{1em}{}

% for customizing subsubsection titles (no new lines after title)
\titleformat{\subsubsection}[runin]{\normalfont\sffamily\bfseries}{\thesubsubsection}{1em}{}

\author{Petr Kotlan}
\title{Optimalizace investičních prostředků z hlediska výnosu fotovoltaických elektráren}
\date{}

\begin{document}

\begin{titlepage}
    \begin{center}
        \vspace*{1cm}
            
        \Huge
        \textbf{\textsf{Univerzita Jana Evangelisty Purkyně \\v Ústí nad Labem}}
            
        \vspace{0.5cm}
        \LARGE
        Přírodovědecká fakulta
        
        \vspace{2cm}
        \includegraphics[width=0.6\textwidth]{static/PřF-UJEP-logo.png}
        \vspace{2cm}
            
        \textbf{Optimalizace investičních prostředků z hlediska výnosu fotovoltaických elektráren}
        
        \vspace{1.5cm}

        BAKALÁŘSKÁ PRÁCE

        \vfill

        \begin{flushleft}
            \textbf{Vypracoval:} Petr Kotlan \\
            \textbf{Vedoucí práce:} Ing. Roman Vaibar, Ph.D., MBA
            
        \end{flushleft}

        \vspace{1.5cm}

        Ústí nad Labem, 2024

    \end{center}
\end{titlepage}

\thispagestyle{empty}
\mbox{}


\includepdf[pages=-]{static/podklad_bp.pdf}
\thispagestyle{empty}
\addtocounter{page}{1} 

\textbf{Prohlášení}

\vspace{1cm}

Prohlašuji, že jsem tuto bakalářskou práci vypracoval samostatně a použil jen pramenů, které
cituji a uvádím v přiloženém seznamu literatury.

\vspace{0.5cm}

Byl jsem seznámen s tím, že se na moji práci vztahují práva a povinnosti vyplývající ze zákona
č. 121/2000 Sb., ve znění zákona č. 81/2005 Sb., autorský zákon, zejména se skutečností,
že Univerzita Jana Evangelisty Purkyně v Ústí nad Labem má právo na uzavření licenční
smlouvy o užití této práce jako školního díla podle § 60 odst. 1 autorského zákona, a s tím, že
pokud dojde k užití této práce mnou nebo bude poskytnuta licence o užití jinému subjektu,
je Univerzita Jana Evangelisty Purkyně v Ústí nad Labem oprávněna ode mne požadovat
přiměřený příspěvek na úhradu nákladů, které na vytvoření díla vynaložila, a to podle
okolností až do jejich skutečné výše.

\vspace{1cm}

\begin{multicols}{2}
    \begin{flushleft}
    V Ústí nad Labem dne \today
    \end{flushleft}

\newcolumn

\begin{flushright}
Podpis: ............................................
\end{flushright}


\end{multicols}

\newpage

\thispagestyle{empty}
\mbox{}
\newpage

\thispagestyle{empty}

% podekovani, vpravo dole
\null
\vfill

\begin{flushright}
    \textit{podekování}
\end{flushright}

\newpage

\thispagestyle{empty}
\mbox{}
\newpage

\thispagestyle{plain}


\textbf{Abstrakt}


\newpage


\thispagestyle{empty}
\mbox{}
\newpage

\tableofcontents

\addcontentsline{toc}{chapter}{Úvod}
\renewcommand{\chaptername}{Úvod}
\chapter*{Úvod}

% \chapter{Současné modely výnosů fotovoltaických elektráren v ČR}
\chapter{Přehled problematiky fotovoltaiky}
\renewcommand{\chaptername}{Přehled problematiky fotovoltaiky}

Úvodní část si klade za cíl seznámit čtenáře s základními pojmy z oblasti fotovoltaiky.

Fotovoltaika je technologie, která přímo přeměňuje sluneční záření na elektřinu.
K výrobě elektřiny není potřeba žádné palivo a elektrárny, proto se fotovoltaické elektrárny řadí mezi obnovitelné zdroje energie.


\section{Technický popis fotovoltaických elektráren}
% https://www.fotovia.cz/komponenty-fve

Při pořizování fotovoltaické elektrárny je k dobrému výběru dobré znát parametry jednotlivých částí, ze kterých se skládá.



\subsection{Veličiny}

\subsubsection{Watt} je jednotkou výkonu a rovná se vykonané práci za jendotku času.
Značíme symbolem \si{\watt}.

\subsubsection{Watt hodina} je jednotkou energie a rovná se práci stroje o výkonu jednoho wattu, který pracuje po dobu jedné hodiny.
Značíme ji symbolem \si{\watt\hour}. Při měření spotřeby elektřiny se nejčastěji se užívá v násobku kilowatt hodiny (\si{\kWh}).

\subsubsection{Killowatt--peak} je jednotkou špičkového výkonu fotovoltaické elektrárny. Tento výkon je při standardních testovacích podmínkách.
Značíme symbolem \si{\kW}p.


\subsection{Komponenty fotovoltaické elektrárny}
To, jaké komponenty se investor rozhodne koupit do svého fotovoltaického systému, může zásadně ovlivnit návratnost této investice.
Některé komponenty jsou pro elektrárnu nezbytné, jiné slouží k optimalizaci výkonu při různých podmínkách či specifickém využití.
Zde uvádím některé základní komponenty fotovoltaické elektrárny.

\subsubsection{Fotovoltaické panely}

jsou nezbytnou součástí každé fotovoltaické elektrárny.
Jejich hlavním úkolem je absorbovat sluneční záření a přeměnit ho na elektrickou energii.

\subsubsection{Baterie}

(nebo také akumulátory) slouží k ukládání přebytků energie, které nebyly spotřebovány.
Dělíme je na virtuální uložiště a fyzické uložiště.

\begin{itemize}
    \item \textbf{Virtuální uložiště} -- funguje na základě podepsání smlouvy s distributorem. Přebytky vyrobené energie se posílají do veřejné sítě, odkud se v případě potřeby mohou odčerpat.  Ve virtuální baterii můžete uložit tolik energie, kolik vám smluvně umožňuje distributor. Mají několik nevýhod:
    \begin{itemize}
        \item je to komerční produkt, který nenabízí všichni distributoři a nikde se negarantují stálé podmínky,
        \item platíte za využívání paušální poplatek,
        \item v případě výpadku elektřiny nemáte záložní zdroj energie.
    \end{itemize}
    \item \textbf{Fyzické bateriové uložiště} -- je zařízení, které je uložené v domě. Vyrobená energie se do baterií ukládá a v případě potřeby se z nich odebírá. Nevýhodou fyzické baterie je její kapacita.
\end{itemize}


% \subsubsection{Virtuální uložiště}
% funguje na základě podepsání smlouvy s distributorem. Přebytky vyrobené energie se posílají do veřejné sítě, odkud se v případě potřeby mohou odčerpat.
% Virtuální úložiště nejsou plnou náhradou fyzických baterií. Mají několik nevýhod:

% \begin{itemize}
%     \item je to komerční produkt, který nenabízí všichni distributoři a nikde se negarantují stálé podmínky,
%     \item platíte za využívání paušální poplatek,
%     \item v případě výpadku elektřiny nemáte záložní zdroj energie. 
% \end{itemize}

% \subsubsection{Fyzické bateriové uložiště} je zařízení, které je uložené v domě. Vyrobená energie se do baterií ukládá a v případě potřeby se z nich odebírá.
% Nevýhodou fyzické baterie je její kapacita. Ve virtuální baterii můžete uložit tolik energie, kolik vám smluvně umožňuje distributor.


\subsubsection{Invertor}

(nebo také měnič či střídač) je podobně jako fotovoltaické panely
nezbytnou součástí každé fotovoltaické elektrárny.
Jeho hlavní funkcí je přeměna stejnosměrného proudu na proud střídavý.
Inverotr rozděluje vyrobenou energii do tří fází. To kolik dávkuje do jednotlivých fází záleží na typu invertoru.

\begin{itemize}
    \item Symetrikcý invertor -- levnější typ, dávkuje elektřinu rovnoměrně do všech tří fází.
    \item Asymetrický invertor -- dražší typ, při dávkování zohledňuje spotřebu na všech fázích.
\end{itemize}

Pokud by tedy například fotovoltaika vyrobila 6 \si{\kWh} elektřiny a spotřeba by na jednotlivých fázích vypdala následovně:

\begin{itemize}
    \item Fáze 1: 3 \si{\kWh}
    \item Fáze 2: 2 \si{\kWh}
    \item Fáze 3: 0.5 \si{\kWh}
\end{itemize}

Symetrický invertor rozdělí elektřinu rovnoměrně (2 \si{\kWh} na každou fázi), takže na fázi 2 a 3 vznikne přebytek, který se pošle do sítě a na fázi 1 energie chybí, tzn. musí se dobrat ze sítě.
Asymetrický invertor elektřinu rozdělí tak, že nebude nutné ze sítě nic dobírat, ani do ní nic posílat.  

\subsubsection{Optimizér}
je zařízení, které se připojuje k jednotlivým fotovoltaickým panelům. Výkon elektrárny se odvíjí od výkony nejméně vákoného článku.
V případě že je snížen výkon jednoho panelu, optimizér zajistí přemostění tohoto panelu.


\subsection{Druhy fotovoltaických systémů}
% https://www.fotovia.cz/blog/typy-fotovoltaickych-elektraren

Rozdílem mezi jednotlivými druhy fotovoltaických systémů je jejich napojení do veřejné elektrické sítě a integrace akumulátorů.
Podle těchto kritérií je lze rozdělit do tří základních kategorií:

\begin{itemize}
    \item ostrovní
    \item standardní
    \item hybridní
\end{itemize}

\subsubsection{Ostrovní elektrárna}

(tzv. off-grid) je samostatný systém, který není připojen k elektrické síti.
Klíčovou částí toho systému je baterie (akumulátor), která slouží k ukládání přebytků energie.
Jsou užitečné v oblastech, kde připojení k elektrické síti není možné.

\begin{multicols}{2}
    \textbf{Výhody:}
    \begin{itemize}[leftmargin=*]
        \item nezávislost na dodavatelích elektřiny,
        \item pokud dojde k výpadku elektřiny, ostrovní elektrárna bude sloužit jako záložní zdroj,
    \end{itemize}
    
    \columnbreak
    
    \textbf{Nevýhody:}
    \begin{itemize}[leftmargin=*]
        \item počáteční náklady mohou být vyšší, kvůli potřebě akumulátorů,
        \item baterie vyžadují pravidelnou údržbu.
    \end{itemize}
\end{multicols}

\begin{figure}[h]
    \centering
    \begin{tikzpicture}[>=stealth,thick]
        % Nodes
        \node[draw, minimum height=1cm] (panel) at (0,0) {Panel};
        \node[draw, circle] (inverter) at (3,0) {Invertor};
        \node[draw] (baterie) at (3,-2) {Baterie};
        \node[draw, circle] (odb) at (6,0) {Spotřebič};
        % Arrows
        \draw[->] (panel) -- (inverter);
        \draw[->] (inverter) -- (baterie);
        \draw[->] (baterie) -- (inverter);
        \draw[->] (inverter) -- (odb);

    \end{tikzpicture}
    \caption{Schéma off-grid fotovoltaické elektrárny}
    \label{fig:offgrid_schema}
\end{figure}

\subsubsection{Standardní elektrárna}

Standardní (tzv. on-grid) fotovoltaická elektrárna je připojena k elektrické síti.
Veškerou přebytečnou energii lze prodat dodavateli elektřiny a v případě potřeby ji lze i odebírat.

\begin{multicols}{2}
    \textbf{Výhody:}
    \begin{itemize}[leftmargin=*]
        \item možnost prodeje přebytků elektřiny,
        \item dlouhá životnost, malá potřeba údržby.
    \end{itemize}
    
    \columnbreak
    
    \textbf{Nevýhody:}
    \begin{itemize}[leftmargin=*]
        \item závislost na síti,
        \item závislost na slunečním záření.
    \end{itemize}
\end{multicols}


\begin{figure}[H]
    \centering
    \begin{tikzpicture}[>=stealth,thick]
        % Nodes
        \node[draw, minimum height=1cm] (panel) at (0,0) {Panel};
        \node[draw, circle] (inverter) at (3,0) {Invertor};
        \node[draw, circle] (odb) at (6,0) {Spotřebič};
        \node[draw, minimum width=2cm] (sita) at (3,-2) {Elektrická síť};
        % Arrows
        \draw[->] (panel) -- (inverter);
        \draw[->] (inverter) -- (odb);
        \draw[->] (inverter) -- (sita);

    \end{tikzpicture}
    \caption{Schéma on-grid fotovoltaické elektrárny}
    \label{fig:ongrid_schema}
    \end{figure}


\subsubsection{Hybridní elektrárna}

kombinuje výhody ostrovních a standardních systémů.
Jsou připojeny k elektrické síti, ale zároveň mají akumulátory, které slouží jako záložní zdroj energie.

\begin{multicols}{2}
    \textbf{Výhody:}
    \begin{itemize}[leftmargin=*]
        \item uložení přebytků energie,
        \item větší energetická nezávislost.
    \end{itemize}
    
    \columnbreak
    
    \textbf{Nevýhody:}
    \begin{itemize}[leftmargin=*]
        \item vysoké počáteční náklady,
        \item baterie vyžadují pravidelnou údržbu.
    \end{itemize}

\end{multicols}

\begin{figure}[h]
    \centering
    \begin{tikzpicture}[>=stealth,thick]
        % Nodes
        \node[draw, minimum height=1cm] (panel) at (0,0) {Panel};
        \node[draw, circle] (inverter) at (3,0) {Invertor};
        \node[draw] (baterie) at (2,-2) {Baterie};
        \node[draw, circle] (odb) at (6,0) {Spotřebič};
        \node[draw, minimum width=2cm] (sita) at (4.5,-2) {Elektrická síť};
        % Arrows
        \draw[->] (panel) -- (inverter);
        \draw[->] (inverter) -- (baterie);
        \draw[->] (baterie) -- (inverter);
        \draw[->] (inverter) -- (odb);
        \draw[->] (inverter) -- (sita);

    \end{tikzpicture}
    \caption{Schéma hybridní fotovoltaické elektrárny}
    \label{fig:hybrid_schema}
\end{figure}


\section{Jak fotovoltaika šetří peníze}
% https://oze.tzb-info.cz/fotovoltaika/24229-stroj-na-penize-fotovoltaika-pri-vysokych-cenach-elektriny-usetri-desetitisice-korun-rocne
Fotovoltaické elektrárny představují významný způsob, jak snižovat náklady na elektřinu v domácnostech.
V domácnosti přispívá fotovoltaika k úspoře peněz dvěma hlavními způsoby.

% \begin{itemize}
%     \item Vyrobená elektřina, která se spotřebuje na místě, se nemusí nakoupit ze sítě.
%     \item Přebytek elektřiny se prodá dodavateli do sítě
% \end{itemize}
\subsection{Spotřeba vyrobené elektřiny}

Vyrobená elektřina, která se spotřebuje na místě, se nemusí nakoupit ze sítě.
Pro tuto úsporu jsou důležité dva faktory:

\begin{itemize}
    \item \textbf{Kolik elektřiny fotovoltaika vyrobí.} V ČR je horní limit pro výkon domácí elektrárny 10 \si{\kW}p. Pokud pro hrubý odhad výroby použijeme jednoduchý přepočet, kdy 1 instalovaný \si{\kW}p vyrobí za rok 1 MWh, zjistíme, že v ČR lze ročně ušetřit maximálně 10 MWh. Na vyšší výkon je potřeba licence nebo provoz v ostrovním režimu.
    \item \textbf{Kolik elektřiny domácnost spotřebuje.} Fotovoltaika vyrábí nejvíce elektřiny vyrábí přes den a v létě. Největší spotřeba domácnosti je ale v zimě a večer. Spotřeba domácnosti
\end{itemize}

% \includegraphics[width=\textwidth]{static/average-daily-consumption.png}
\begin{figure}
    % https://www.researchgate.net/publication/326358349_Load_Profile_of_Typical_Residential_Buildings_in_Bulgaria
    \includegraphics[width=\textwidth]{static/average-daily-consumption.png}
    \caption{Denní spotřeba elektřiny v domácnosti během všedního dne.} % Zdroj: \cite{bulgaria}
    \label{fig:average_daily_consumption}
\end{figure}

\subsection{Prodej přebytků elektřiny}


\chapter{Teoretická část}
\renewcommand{\chaptername}{Teoretická část}

Tato kapitola je rozdělena do dvou částí. První část se zabývá základními ekonomickými pojmy využívanými v investiční analýze.
Druhá část se zabývá matematickou optimalizací metodou lineárního programování.

\section{Přehled ekonomických pojmů}

\subsection{Základní pojmy}

\subsection{Ukazatele výnosnosti investice}


\subsubsection{Návratnost investice}
($ROI$ -- Return of Investment)
% https://www.moneta.cz/slovnik-pojmu/detail/roi
vyjadřuje zisk nebo ztrátu z investice v procentech.

\begin{equation}
    ROI = \frac{(P_1 + P_2 + \ldots + P_n) - K}{K} \cdot 100,
\end{equation}
kde
\begin{itemize}[label={}]
    \item $n$ -- počet let,
    \item $P_1, P_2, \ldots, P_n$ -- peněžní příjmy z investice v jednotlivých letech,
    \item $K$ -- kapitálový výdaj,
    \item $ROI$ -- návratnost investice.
\end{itemize}

\subsubsection{Diskontované cash-flow}
($DCF$ -- Discounted Cash Flow)
vyjadřuje současnou hodnotu budoucích peněžních toků.

\begin{equation}
    DCF = \frac{P_1}{(1+i)} + \frac{P_2}{(1+i)^2} + \ldots + \frac{P_n}{(1+i)^n},
\end{equation}

kde

\begin{itemize}[label={}]
    \item $n$ -- počet let,
    \item $P_1, P_2, \ldots, P_n$ -- peněžní příjmy z investice v jednotlivých letech,
    \item $i$ -- úroková míra (diskontní sazba),
    \item $DCF$ -- diskontované cash-flow.
\end{itemize}

\subsubsection*{Čistá současná hodnota}
($NPV$ -- Net Present Value)
vyjadřuje současnou hodnotu budoucích peněžních toků po odečtení kapitálového výdaje.

\begin{equation}    
NPV = \frac{P_1}{(1+i)} + \frac{P_2}{(1+i)^2} + \ldots + \frac{P_n}{(1+i)^n} - K,
\end{equation}

\subsubsection*{Vnitřní výnosové procento}
($IRR$ -- Internal Rate of Return)
je úroková míra, při níž se současná hodnota peněžních příjmů z investice rovná kapitálovým výdajům. Investice se považuje za výhodnou, když $IRR$ představuje vyšší úrok, než je požadovaná minimální výnosnost investice.

\begin{equation}
    \frac{P_1}{(1+IRR)} + \frac{P_2}{(1+IRR)^2} + \ldots + \frac{P_n}{(1+IRR)^n} = K,
\end{equation}

kde
\begin{itemize}[label={}]
    \item $n$ -- počet let,
    \item $P_1, P_2, \ldots, P_n$ -- peněžní příjmy z investice v jednotlivých letech,
    \item $K$ -- kapitálový výdaj,
    \color{red}
    \item $IRR$ -- požadovaná míra výnosnosti. NEDAVA SMYSL
\end{itemize}


% \section{Základní modely matematické optimalizace}
\subsection{Formulace úlohy lineárního programování}

\subsubsection{Účelová funkce}


Účelová funkce je lineární funkcí $n$ proměnných ve tvaru

$$ z = c_1x_1 + c_2x_2 + \ldots + c_nx_n , $$


kde $c_1, c_2, \ldots c_n$ jsou konstanty, které nazýváme \textit{cenové koeficienty} nebo \textit{koeficienty účelové funkce} a 
$x_1, x_2, \ldots x_n$ jsou \textit{strukturní neznámé}.

Účelová funkce se buď maximalizuje

$$ \max z = c_1x_1 + c_2x_2 + \ldots + c_nx_n ,$$

nebo minimalizuje

$$ \min z = c_1x_1 + c_2x_2 + \ldots + c_nx_n .$$

\subsubsection{Omezující podmínky}

Omezující podmínky jsou lineární rovnice nebo nerovnice ve tvaru

$$ a_{11}x_1 + a_{12}x_2 + \ldots + a_{1n}x_n \leq b_1 ,$$
$$ a_{21}x_1 + a_{22}x_2 + \ldots + a_{2n}x_n \leq b_2 ,$$
$$ \vdots $$
$$ a_{m1}x_1 + a_{m2}x_2 + \ldots + a_{mn}x_n \leq b_m ,$$

nebo

$$ a_{11}x_1 + a_{12}x_2 + \ldots + a_{1n}x_n \geq b_1 ,$$
$$ a_{21}x_1 + a_{22}x_2 + \ldots + a_{2n}x_n \geq b_2 ,$$
$$ \vdots $$
$$ a_{m1}x_1 + a_{m2}x_2 + \ldots + a_{mn}x_n \geq b_m ,$$

kde $a_{ij}$ jsou konstanty, které nazýváme \textit{strukturní koeficienty} nebo \textit{koeficienty omezení}, $b_i$ jsou konstanty (tzv. \textit{požadavková čísla})
a $x_1, x_2, \ldots x_n$ jsou \textit{strukturní neznámé}.

Zároveň omezující podmínky vymezují pro každou proměnnou $x_1, x_2, \ldots x_n$ množinu hodnot, kterýh může nabývat. 
Nejčastěji se jedná o podmínky tvaru $x_i \geq 0$ (nezápornost).
Jinými případy mohou být například podmínky tvaru $x_i \leq 0$ (nekladnost) nebo $x_i$ může nabývat libovolné hodnoty („neomezeno“).

\subsection{Maticové vyjádření}

% Můžeme vyjádřit \textit{cenové koeficienty} jako vektor $c = (c_1, c_2, \ldots, c_n)^T$.
Můžeme vyjádřit účelovou funkci jako $$ z = c^Tx ,$$
kde $c = (c_1, c_2, \ldots, c_n)^T$ je vektor cenových koeficientů a $x = (x_1, x_2, \ldots, x_n)^T$ je vektor strukturních neznámých.

Omezující podmínky můžeme vyjádřit jako maticový součin

$$ Ax \leq b ,$$

kde $A$ je matice strukturních koeficientů a $b$ je vektor pravých stran omezujících podmínek.

\subsection{Typy úloh lineárního programování}

\chapter{Praktická část}
\renewcommand{\chaptername}{Praktická část}

\section{Popis aplikace}
\subsection{Data}

\subsubsection*{Český hydrometeorologický ústav}

\textbf{ČHMÚ}

\href{https://www.chmi.cz/files/portal/docs/meteo/ok/open_data_2023/Podminky_uziti_udaju.pdf}{Podmínky užití dat}

\subsubsection*{OTE, a.s.}

OTE (Otevřený trh s elektřinou)



\section{Případové studie}

\renewcommand{\chaptername}{Zhodnocení výsledků a závěr}
\chapter{Zhodnocení výsledků a závěr}

\addcontentsline{toc}{chapter}{Seznam zdrojů}
\bibliographystyle{plain}
\begin{thebibliography}{9}
    \bibitem{ote}
    \textit{Krátkodobé trhy}. Online. OTE. C2018. Dostupné z: \url{https://www.ote-cr.cz/cs/kratkodobe-trhy/elektrina/vnitrodenni-trh}. [cit. 2023-12-03].
    \bibitem{matematika_pro_ekonomy}
    STOLÍN, Radek. \textit{Matematika pro ekonomy}. 2., upr. vyd. Jihlava: Vysoká škola polytechnická Jihlava, 2011. ISBN ISBN978-80-87035-35-1.
    \bibitem{demel}
    DEMEL, Jiří. \textit{Operační výzkum}.
\end{thebibliography}

\end{document}