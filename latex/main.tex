\documentclass[12pt, aspectratio=169]{beamer} % options: handout

\usetheme{metropolis}
\setbeamertemplate{frame footer}{\insertshorttitle~| \insertsection}
\setbeamerfont{page number in head/foot}{size=\tiny}
\setbeamercolor{footline}{fg=gray}

\usepackage[greek, czech]{babel}
\usepackage[T1]{fontenc}
\usepackage{amssymb}
\usepackage{amsmath}
\usepackage{amssymb}

\metroset{sectionpage=none} % schova stranky s nazvem sekce

\title[Optimalizace FVE]{Optimalizace investičních prostředků z hlediska výnosu fotovoltaických elektráren}
\author[Kotlan]{Petr Kotlan}
\institute[Přf UJEP]{Přírodovědecká fakulta\\ Univerzita J. E. Purkyně}
\date{}

\begin{document}


\begin{frame}[plain]

    \maketitle
    
\end{frame}

\section{Anotace}

\begin{frame}{\insertsection}
Cílem bakalářské práce je vyvinout aplikaci, která pomocí lineárního programování
optimalizuje rozdělení investičních prostředků pro instalaci fotovoltaických elektráren na daných objektech.
Optimalizace bude provedena na základě následujících hledisek:
\begin{itemize}

    \item Typu střechy – rovná, sedlová, valbová, atd.,
    \item spotřeby v daném místě,
    \item ceně energie definované odkupem dle spotových cen OTE,
    \item optimalizace uložiště,
    \item výpočet předpokládaného ročního výkonu dle osvitových hodin.

\end{itemize}
\end{frame}

\section{Osnova}

\begin{frame}{\insertsection}
    \begin{enumerate}
        \item Úvod
        \item Současné modely výnosů fotovoltaických elektráren v ČR
        \item Teoretická část
        \begin{itemize}
            \item Přehled ekonomických pojmů
            \item Základní modely matematické optimalizace
        \end{itemize}
        \item Praktická část
        \begin{itemize}
            \item Popis aplikace
            \item Případové studie
        \end{itemize}
        \item Zhodnocení výsledků
        \item Závěr
        
    \end{enumerate}
\end{frame}

\section{Lineární programování}

    \begin{frame}{\insertsection}
    \begin{block}{Účelová funkce}
        \vspace{10pt}
        \centering
        Min $c_1x_1 + c_2x_2 + \ldots + c_nx_n$
    \end{block}

    \begin{block}{Omezující podmínky}
        \vspace{10pt}
        \centering
        $Ax \leq b$
        \break
        $x \geq 0$
    \end{block}

    \end{frame}

\section{Porovnání}

\begin{frame}{\insertsection}
    \begin{itemize}
        \item dluhopisy
        \item \href{https://www.mesec.cz/produkty/sporici-ucty/}{spořící účty}
        \item NPV – čistá současná hodnota
        \item IRR – vnitřní výnosové procento
    \end{itemize}

\end{frame}
\section{Čistá současná hodnota}

\begin{frame}{\insertsection}

    \begin{center}
        $NPV = \frac{P_1}{(1+i)} + \frac{P_2}{(1+i)^2} + \ldots + \frac{P_N}{(1+i)^N} - K$
    \end{center}
    
    \vspace{10pt} % Adjust the space here
    \begin{itemize}
        \item $NPV$ = čistá současná hodnota
        \item $P_n$ = peněžní příjem z investice v jednotlivých letech její životnosti
        \item $i$ = požadovaná výnosnost
        \item $N$ = doba životnosti
        \item $K$ = kapitálový výdaj
    \end{itemize}

\end{frame}

\section{Vnitřní výnosové procento}

\begin{frame}{\insertsection}

    \begin{center}
         $\frac{P_1}{(1+i)} + \frac{P_2}{(1+i)^2} + \ldots + \frac{P_N}{(1+i)^N} = K$
    \end{center}
    
    \vspace{10pt}

    \begin{itemize}
        \item $IRR$ = vnitřní výnosové procento
        \item $P_n$ = peněžní příjem z investice v jednotlivých letech její životnosti
        \item $N$ = doba životnosti
        \item $K$ = kapitálový výdaj
        \item $i$ = hledaný úrokový koeficient
    \end{itemize}

\end{frame}

\end{document}