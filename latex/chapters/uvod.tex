Energie jsou z ekonomického hlediska nezanedbatelnou složkou.
V posledních letech, kdy ceny energií rostou, je více než žádoucí jakákoliv příležitost finanční úlevy.

V současné době je fotovoltaika v oblasti energetiky jedním z nejvíce diskutovaných témat.
\textit{motivace, proc problem resit}

\textit{optimalizace pomoci lineárního programování}

\textit{záměr práce, komu je práce určena?}
Cílem této bakalářské práce je seznámit čtenáře s oblastí fotovoltaiky a jejím potenciálem finanční úspory.
Záměrem úvodní části je úvést přehled základních informací o fotovoltaice a jejích výhodách.
Záměrem teoretické části je nejdřive přiblížit obecně teorii investičního rozhodování a následně představit základní formulaci a jednoduché modely lineárního programování.

Práce je členěna do tří částí.
První část se věnuje samotné fotovoltaice.
Obsahuje základní technický popis, druhy různých systémů, které se využívají v praxi, a objasňuje proces, který vede k úsporám.
Následující teoretická část obsahuje pojmy a ukazatele z oblasti investičního rozhodování a teoretický základ z oblasti lineárního programování.
Matematický základ je potom demonstrován na typických a jednoduchých příkladech.
Závěrečná část práce je věnována popisu aplikace, která pro kontkrétní vstup aplikuje teoretické poznatky na simulovaných datech.
