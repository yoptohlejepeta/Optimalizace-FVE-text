\section{Komponenty fotovoltaické elektrárny}

Fotovoltaická elektrárna se skládá z několika základních komponent.
\subsection{Fotovoltaický panel}

\subsection{Baterie}

\subsection{Invertor}



\section{Druhy fotovoltaických systémů}
% https://www.fotovia.cz/blog/typy-fotovoltaickych-elektraren

Rozdílem mezi jednotlyvími druhy fotovoltaických systémů je jejich napojení do veřejné elektrické sítě a integrace akumulátorů.
Podle těchto kritérií je lze rozdělit do tří základních kategorií:

\begin{itemize}
    \item ostrovní
    \item standardní
    \item hybridní
\end{itemize}

\subsection{Ostrovní elektrárna}

Ostrovní (tzv. off-grid) fotovoltaická elektrárna je samostatný systém, který není připojen k elektrické síti.
Klíčovou částí toho systému je baterie (akumulátor), která slouží k ukládání přebytků energie.
Jsou užitečné v oblastech, kde připojení k elektrické síti není možné.

\newpage

\begin{multicols}{2}
    \textbf{Výhody:}
    \begin{itemize}[leftmargin=*]
        \item nezávislost na dodavatelích elektřiny,
        \item pokud dojde k výpadku elektřiny, ostrovní elektrárna bude sloužit jako záložní zdroj,
    \end{itemize}
    
    \columnbreak
    
    \textbf{Nevýhody:}
    \begin{itemize}[leftmargin=*]
        \item počáteční náklady mohou být vyšší, kvůli potřebě akumulátorů,
        \item baterie vyžadují pravidelnou údržbu.
    \end{itemize}
\end{multicols}

\subsection{Standardní elektrárna}

Standardní (tzv. on-grid) fotovoltaická elektrárna je připojena k elektrické síti.
Veškerou přebytečnou energii lze prodat dodavateli elektřiny.

\begin{multicols}{2}
    \textbf{Výhody:}
    \begin{itemize}[leftmargin=*]
        \item možnost prodeje přebytků elektřiny,
        \item dlouhá životnost, malá potřeba údržby.
    \end{itemize}
    
    \columnbreak
    
    \textbf{Nevýhody:}
    \begin{itemize}[leftmargin=*]
        \item závislost na síti,
        \item závislost na slunečním záření.
    \end{itemize}
\end{multicols}


\subsection{Hybridní elektrárna}

Hybridní fotovoltaické elektrárny kombinují výhody ostrovních a standardních systémů.
Jsou připojeny k elektrické síti, ale zároveň mají akumulátory, které slouží jako záložní zdroj energie.

\begin{multicols}{2}
    \textbf{Výhody:}
    \begin{itemize}[leftmargin=*]
        \item uložení přebytků energie,
        \item větší energetická nezávislost.
    \end{itemize}
    
    \columnbreak
    
    \textbf{Nevýhody:}
    \begin{itemize}[leftmargin=*]
        \item vysoké počáteční náklady,
        \item baterie vyžadují pravidelnou údržbu.
    \end{itemize}
    
\end{multicols}