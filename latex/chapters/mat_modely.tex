\section{Lineární programování}
Tato část vychází ze dvou učebních textů. Prvním je \textit{Matematika pro ekonomy} od R. Stolína \cite{matematika_pro_ekonomy} a druhým je \textit{Operační výzkum} od J. Demela \cite{demel}.
% TODO: 3 kniha

\subsection{Formulace úlohy lineárního programování}

Lineární programování patří k metodám \textit{operačního výzkumu}.
% Je zaměřeno na hledání optimálního řešení při kterém, jsou zároveň splněny omezující podmínky.
Cílem je nalezení optimálního řešení při splnění daných podmínek.

\subsubsection{Účelová funkce}

je lineární funkcí $n$ proměnných ve tvaru

\begin{equation} \label{objective}
    z = c_1x_1 + c_2x_2 + \ldots + c_nx_n ,
\end{equation}


kde $c_1, c_2, \ldots c_n$ jsou konstanty, které nazýváme \textit{cenové koeficienty} nebo \textit{koeficienty účelové funkce} a 
$x_1, x_2, \ldots x_n$ jsou \textit{strukturní neznámé}.

Účelová funkce se buď maximalizuje

\begin{equation} \label{max_objective}
    \max z = c_1x_1 + c_2x_2 + \ldots + c_nx_n ,
\end{equation}

nebo minimalizuje

\begin{equation} \label{min_objective}
    \min z = c_1x_1 + c_2x_2 + \ldots + c_nx_n .
\end{equation}

\subsubsection{Omezující podmínky}

jsou lineární rovnice nebo nerovnice ve tvaru

\begin{equation} \label{constraints}
    \begin{gathered}
        a_{11}x_1 + a_{12}x_2 + \ldots + a_{1n}x_n \ \lesseqgtr \ b_1, \\
        a_{21}x_1 + a_{22}x_2 + \ldots + a_{2n}x_n \ \lesseqgtr \ b_2, \\
        \vdots \\
        a_{m1}x_1 + a_{m2}x_2 + \ldots + a_{mn}x_n \ \lesseqgtr \ b_m,
    \end{gathered}
\end{equation}

kde na místě označeném $\lesseqgtr$ se může vyskytnout symbol $\leq$, $\geq$ nebo $=$\footnote{Tato formulace je doslovným přepisem z \textit{Operační výzkum} J. Demel \cite{demel} str. 9.}.

Prvky $a_{ij}$ jsou konstanty, které nazýváme \textit{strukturní koeficienty} nebo \textit{koeficienty omezení},
$b_1, b_2, \ldots, b_m$
jsou konstanty (tzv. \textit{požadavková čísla})
jsou konstanty, které nazýváme \textit{strukturní koeficienty} nebo \textit{koeficienty omezení}, $b_i$ jsou konstanty (tzv. \textit{požadavková čísla})
a $x_1, x_2, \ldots x_n$ jsou \textit{strukturní neznámé}.

Zároveň omezující podmínky vymezují pro každou proměnnou $x_1, x_2, \ldots x_n$ množinu hodnot, kterých může nabývat. 
Jedná o podmínky tvaru
\begin{equation} \label{sign_restrictions}
    x_i \geq 0 \quad \text{(nezápornost)} \quad \text{nebo} \quad x_i \leq 0 \quad \text{(nekladnost)}.
\end{equation}

Jinými případ může být, kdy $x_i$ může nabývat libovolné hodnoty („neomezeno“).

\subsubsection{Přípustným řešením} se nazývá vektor $x = (x_1, x_2, \ldots, x_n)$, který vyhovuje všem omezujícím podmínkám (\ref{constraints}) a zároveň podmínkám nezápornosti respektive nekladnosti (\ref{sign_restrictions})

\subsubsection{Optimální řešení} úlohy, je takové přípustné řešení, kdy účelová funkce nabývá maxima (\ref{max_objective}) respektive minima (\ref{min_objective}).

\subsection{Maticový zápis úlohy LP}

Celý problém lineárního programování můžeme pro přehlednost zapsat maticově.
Účelovou funkci vyjádříme jako
jako

\begin{equation}
    z = \bm{c}^T\bm{x} \rightarrow \max,
\end{equation}

nebo

\begin{equation}
    z = \bm{c}^T\bm{x} \rightarrow \min,
\end{equation}

kde $\bm{c}$ je sloupcový vektor cenových koeficientů a $\bm{x}$ je sloupcový vektor \textit{strukturních neznámých}.

\textit{Strukturní koeficienty} můžeme vyjádřit jako matici

\begin{equation}
    \bm{A} = 
    \begin{pmatrix}
        a_{11} & a_{12} & \ldots & a_{1n} \\
        a_{21} & a_{22} & \ldots & a_{2n} \\
        \vdots & \vdots & \ddots & \vdots \\
        a_{m1} & a_{m2} & \ldots & a_{mn}
    \end{pmatrix}
\end{equation}

a za předpokladu, že jsou všechny omezující podmínky stejného typu (tzn. $\leq$, $\geq$ nebo $=$), můžeme je vyjádřit jako 

\begin{equation}
    \begin{split}
        \bm{A}\bm{x} &\leq \bm{b} \quad \text{nebo} \quad 
        \bm{A}\bm{x} \geq \bm{b} \quad \text{nebo} \quad 
        \bm{A}\bm{x} = \bm{b},
    \end{split}
    \end{equation}

kde $\bm{b}$ je sloupcový vektor \textit{požadavkových čísel}.

\subsection{Typy úloh lineárního programování}

V následující časti jsou popsány některé typické úlohy lineárního programování.
% \footnote{Podle \textit{Matematika pro ekonomy} R. Stolín \cite{matematika_pro_ekonomy}}

\subsubsection{Úlohy výrobního plánování}
jsou jedny z nejčastěji řešených úloh lineárního programování.
Představme si výrobce, který má možnost vyrábět $n$ druhů produktů.
Má k dispozici omezenou kapacitu $m$ výrobních zdrojů (pracovníci, stroje, suroviny, atd.).
Cílem je určit, jaký objem jednotlivých produktů má být vyroben, aby bylo dosaženo maximálního zisku.

Pojďmě si tedy říct, co pro nás proměnné z obecného vyjádření úlohy lineárního programování znamenají konkrétně v případě výrobního plánování.

\begin{itemize}[label={}]
    \item $x_j$ ($j = 1, 2, \ldots, n$) je množství $j$-tého produktu, které má být vyrobeno \\(tzn. \textit{strukturní neznámá}),
    \item $c_j$ ($j = 1, 2, \ldots, n$) je zisk z prodeje jedné jednotky $j$-tého produktu \\(tzn. \textit{cenový koeficient}),
    \item $a_{ij}$ ($i = 1, 2, \ldots, m$; $j = 1, 2, \ldots, n$) je množství $i$-tého výrobního zdroje, které je potřeba k výrobě jedné jednotky $j$-tého produktu \\(tzn. \textit{strukturní koeficient}),
    \item $b_i$ ($i = 1, 2, \ldots, m$) je množství $i$-tého výrobního zdroje, které je k dispozici \\(tzn. \textit{požadavkové číslo}).
\end{itemize}

Úloha má pak tvar

\begin{equation}
        \max z = c_1x_1 + c_2x_2 + \ldots + c_nx_n,
\end{equation}

za podmínek

\begin{equation}
    \begin{gathered}
        a_{11}x_1 + a_{12}x_2 + \ldots + a_{1n}x_n \leq b_1, \\
        a_{21}x_1 + a_{22}x_2 + \ldots + a_{2n}x_n \leq b_2, \\
        \vdots \\
        a_{m1}x_1 + a_{m2}x_2 + \ldots + a_{mn}x_n \leq b_m, \\
        x_1, x_2, \ldots, x_n \geq 0.
    \end{gathered}
\end{equation}

Ve zkráceném zápisu

\begin{equation}
    \begin{gathered}
        \max z = \bm{c}^T\bm{x}, \\
        \bm{A}\bm{x} \leq \bm{b}, \\
        \bm{x} \geq 0.
    \end{gathered}
\end{equation}

\textbf{Příklad:}

Podnik vyrábí dva druhy produktů $A$ a $B$. Každý z nich vyžaduje určité množství surovin na výrobu.
Na vyrobení jednoho kusu produktu $A$ je potřeba 2 jednotky suroviny $S_1$, 1 jednotky suroviny $S_2$ a 1 jednotka suroviny $S_3$.
Na vyrobení jednoho kusu produktu $B$ je potřeba 2 jednotky suroviny $S_1$, 3 jednotky suroviny $S_2$ a 1 jednotka suroviny $S_3$.
Celkové množství surovin $S_1$, $S_2$ a $S_3$ je omezeno na 40, 30 a 20 jednotek.
Zisk z prodeje jednoho kusu produktu $A$ je 5 Kč a z prodeje jednoho kusu produktu $B$ je 10 Kč.

Sestavíme přehlednou tabulku se všemi informacemi.

\begin{table}[H]
    \centering
    \begin{tabular}{|c|c|c|c|}
        \hline
        \textbf{} & \textbf{A} & \textbf{B} & \textbf{Disponibilní množství (ks)}\\
        \hhline{|=|=|=|=|}
        $S_1$ & 2 & 2 & 40 \\
        \hline
        $S_2$ & 1 & 3 & 30 \\
        \hline
        $S_3$ & 1 & 1 & 20 \\
        \hhline{|=|=|=|=|}
        \textbf{Zisk (Kč)} & 5 & 10 & -- \\
        \hline
    \end{tabular}
    \caption{Informace o produktech a surovinách.}
    \label{tab:vyroba}
\end{table}

Nyní můžeme sestavit úlohu matematcký model.

\begin{equation}
    \begin{gathered}
        \max z = 5x_1 + 10x_2, \\
        2x_1 + 2x_2 \leq 40, \\
        x_1 + 3x_2 \leq 30, \\
        x_1 + x_2 \leq 20, \\
        x_1, x_2 \geq 0.
    \end{gathered}
\end{equation}

\subsubsection{Směšovací úlohy}
mají za úkol z $m$ daných komponent sestavit směs, která musí obsahovat $n$ různých látek za co nejnižší cenu.
Známe složení a cenu jednotlivých komponent.

Proměnné vyjádříme následovně:

\begin{itemize}[label={}]
    \item $x_j$ ($j = 1, 2, \ldots, m$) je množství $j$-té komponenty ve směsi \\(tzn. \textit{strukturní neznámá}),
    \item $c_j$ ($j = 1, 2, \ldots, m$) je cena jedné jednotky $j$-té komponenty \\(tzn. \textit{cenový koeficient}),
    \item $a_{ij}$ ($i = 1, 2, \ldots, n$; $j = 1, 2, \ldots, m$) je obsah $i$-té látky v jednotce $j$-té komponenty \\(tzn. \textit{strukturní koeficient}),
    \item $b_i$ ($i = 1, 2, \ldots, n$) je požadované množství $i$-té látky v směsi \\(tzn. \textit{požadavkové číslo}).
\end{itemize}

Úloha má pak tvar

\begin{equation}
    \begin{gathered}
        \min z = c_1x_1 + c_2x_2 + \ldots + c_mx_m, \\
        a_{11}x_1 + a_{12}x_2 + \ldots + a_{1m}x_m = b_1, \\
        a_{21}x_1 + a_{22}x_2 + \ldots + a_{2m}x_m = b_2, \\
        \vdots \\
        a_{n1}x_1 + a_{n2}x_2 + \ldots + a_{nm}x_m = b_n, \\
        x_1, x_2, \ldots, x_m \geq 0.
    \end{gathered}
\end{equation}

Ve zkráceném zápisu

\begin{equation}
    \begin{gathered}
        \min z = \bm{c}^T\bm{x}, \\
        \bm{A}\bm{x} = \bm{b}, \\
        \bm{x} \geq 0.
    \end{gathered}
\end{equation}

\textbf{Příklad:}

K dispozici máme šest druhů potravin ($P_1$ až $P_6$), ze kterých je třeba sestavit denní jídelníček.
Jídelníček musí obsahovat alespoň \SI{3000}{kcal} využitelné energie,
\SI{70}{g} bílkovin,
\SI{1,8}{mg} vitamínu B,
\SI{75}{mg} vitamínu C a zároveň být co nejlevnější.
V tabulce \ref{tab:potraviny} jsou uvedeny údaje o obsahu látek v jednotce přislušné potraviny a cena jedné jednotky.
\footnote{Hodnoty jsou převzaty z \textit{Matematika pro ekonomy} R. Stolín \cite{matematika_pro_ekonomy} příklad 6.2.1}
\begin{table}[H]
    \centering
    \begin{tabular}{|c|c|c|c|c|c|c|}
        \hline
        \textbf{} & \textbf{P1} & \textbf{P2} & \textbf{P3} & \textbf{P4} & \textbf{P5} & \textbf{P6} \\
        \hhline{|=|=|=|=|=|=|=|}
        Energie (\si{kcal}) & 2350 & 3540 & 700 & 2830 & 370 & 200\\
        \hline
        Bílkoviny (\si{g}) & 61 & 67 & 16 & 216 & -- & 25\\
        \hline
        Vitamin B (\si{mg}) & 1,1 & 0,2 & 0,7 & 8,1 & 0,3 & 1,3\\
        \hline
        Vitamin C (\si{mg}) & -- & -- & 100 & -- & 600 & 630\\
        \hline
        Cena (Kč/j) & 26 & 50 & 80 & 50 & 100 & 15\\
        \hline
    \end{tabular}
    \caption{Informace o potravinách.}
    \label{tab:potraviny}
\end{table}

Matematický model bude vypadat následovně.

% TODO: dodelat

\begin{equation*}
    \max z = 26x_1 + 50x_2 + 80x_3 + 50x_4 + 100x_5 + 15x_6,
\end{equation*}


\begin{equation}
    \begin{aligned}
    2350x_1 &+ &3540x_2 &+ &700x_3 &+ &2830x_4 &+ &370x_5 &+ &200x_6 &= 3000, \\
    61x_1 &+ &67x_2 &+ &16x_3 &+ &216x_4 && &+ &25x_6 &= 70, \\
    1.1x_1 &+ &0.2x_2 &+ &0.7x_3 &+ &8.1x_4 && &+ &1.3x_6 &= 1.8, \\
    &&&& 100x_3 && &+ &600x_5 &+ &630x_6 &= 75, \\
    \end{aligned}
\end{equation}

\begin{equation*}
    x_1, x_2, x_3, x_4, x_5, x_6 \geq 0.
\end{equation*}


\subsubsection{Dopravní úlohy} mají od předešlích typů úloh odlišnou strukturu.
Kvůli tomu se také vyvinuly jiné a efektivnější metody řešení, proto si pouze ukážeme jejich matematickou formulaci a nebudeme se jimi zabývat podrobněji.

Máme $m$ dodavatelů, kteří dodávají stejný druh zbtoží, a $n$ odběratelů tohoto zboží.
Jsou nám známy ceny za dopravu mezi jednotlivými dodavateli a odběrateli.
Cílem je minimalizovat náklady na dopravu zboží. Proměnné vyjádříme následovně:

\begin{itemize}[label={}]
    \item $x_{ij}$ ($i = 1, 2, \ldots, m$; $j = 1, 2, \ldots, n$) je množství zboží dopravované od $i$-tého dodavatele k $j$-tému odběrateli,
    \item $c_{ij}$ ($i = 1, 2, \ldots, m$; $j = 1, 2, \ldots, n$) je cena za přepravu jedné jednotky zboží od $i$-tého dodavatele k $j$-tému odběrateli,
    \item $a_i$ ($i = 1, 2, \ldots, m$) kapacita zboží $i$-tého dodavatele,
    \item $b_j$ ($j = 1, 2, \ldots, n$) je množství zboží, které $j$-tý odběratel požaduje.
\end{itemize}

Úloha má pak tvar

\begin{equation}
    \begin{gathered}
        \min z = \sum_{i=1}^{m} \sum_{j=1}^{n} c_{ij}x_{ij}, \\
        \sum_{j=1}^{n} x_{ij} \leq a_i \quad (i = 1, 2, \ldots, m), \\
        \sum_{i=1}^{m} x_{ij} \geq b_j \quad (j = 1, 2, \ldots, n), \\
        x_{ij} \geq 0 \quad (i = 1, 2, \ldots, m; j = 1, 2, \ldots, n).
    \end{gathered}
\end{equation}

