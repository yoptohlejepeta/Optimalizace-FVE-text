\subsection{Formulace úlohy lineárního programování}

\subsubsection{Účelová funkce}


Účelová funkce je lineární funkcí $n$ proměnných ve tvaru

\begin{equation}
    z = c_1x_1 + c_2x_2 + \ldots + c_nx_n ,
\end{equation}


kde $c_1, c_2, \ldots c_n$ jsou konstanty, které nazýváme \textit{cenové koeficienty} nebo \textit{koeficienty účelové funkce} a 
$x_1, x_2, \ldots x_n$ jsou \textit{strukturní neznámé}.

Účelová funkce se buď maximalizuje

\begin{equation}
    \max z = c_1x_1 + c_2x_2 + \ldots + c_nx_n ,
\end{equation}

nebo minimalizuje

\begin{equation}
    \min z = c_1x_1 + c_2x_2 + \ldots + c_nx_n .
\end{equation}

\subsubsection{Omezující podmínky}

Omezující podmínky jsou lineární rovnice nebo nerovnice ve tvaru

% TODO: ocitovat doslovne prepisy
\begin{equation}
    \begin{gathered}
        a_{11}x_1 + a_{12}x_2 + \ldots + a_{1n}x_n \ \lesseqgtr \ b_1 \\
        a_{21}x_1 + a_{22}x_2 + \ldots + a_{2n}x_n \ \lesseqgtr \ b_2 \\
        \vdots \\
        a_{m1}x_1 + a_{m2}x_2 + \ldots + a_{mn}x_n \ \lesseqgtr \ b_m
    \end{gathered}
\end{equation}

kde na místě $\lesseqgtr$ může být $\leq$, $\geq$ nebo $=$.

Prvky $a_{ij}$ jsou konstanty, které nazýváme \textit{strukturní koeficienty} nebo \textit{koeficienty omezení},
$b_1, b_2, \ldots, b_m$
jsou konstanty (tzv. \textit{požadavková čísla})
jsou konstanty, které nazýváme \textit{strukturní koeficienty} nebo \textit{koeficienty omezení}, $b_i$ jsou konstanty (tzv. \textit{požadavková čísla})
a $x_1, x_2, \ldots x_n$ jsou \textit{strukturní neznámé}.

Zároveň omezující podmínky vymezují pro každou proměnnou $x_1, x_2, \ldots x_n$ množinu hodnot, kterýh může nabývat. 
Nejčastěji se jedná o podmínky tvaru $x_i \geq 0$ (nezápornost).
Jinými případy mohou být například podmínky tvaru $x_i \leq 0$ (nekladnost) nebo $x_i$ může nabývat libovolné hodnoty („neomezeno“).

\subsection{Maticové vyjádření}

Můžeme vyjádřit účelovou funkci jako $$ z = \bm{c}^T\bm{x} \rightarrow \max ,$$
kde $\bm{c} = (c_1, c_2, \ldots, c_n)^T$ je vektor cenových koeficientů a $\bm{x} = (x_1, x_2, \ldots, x_n)^T$ je vektor strukturních neznámých.

Omezující podmínky můžeme vyjádřit jako maticový součin

$$ \bm{A}\bm{x} \leq \bm{b} ,$$

kde $\bm{A}$ je matice strukturních koeficientů, $\bm{x}$ je vektor strukturních neznámých
a $\bm{b}$ je vektor pravých stran omezujících podmínek.

\subsection{Typy úloh lineárního programování}