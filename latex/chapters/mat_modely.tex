\section{Matematická optimalizace}
Tato část vychází ze dvou učebních textů. Prvním je \textit{Matematika pro ekonomy} od R. Stolína \cite{matematika_pro_ekonomy} a druhým je \textit{Operační výzkum} od J. Demela \cite{demel}.


V úvodu této kapitoly jsou popsány základní pojmy a formulace úlohy lineárního programování.

Lineární programování patří k metodám \textit{operačního výzkumu}.
Je zaměřeno na hledání optimálního řešení při kterém, jsou zároveň splněny omezující podmínky.

\subsection{Formulace úlohy lineárního programování}

\subsubsection{Účelová funkce}

je lineární funkcí $n$ proměnných ve tvaru

\begin{equation}
    z = c_1x_1 + c_2x_2 + \ldots + c_nx_n ,
\end{equation}


kde $c_1, c_2, \ldots c_n$ jsou konstanty, které nazýváme \textit{cenové koeficienty} nebo \textit{koeficienty účelové funkce} a 
$x_1, x_2, \ldots x_n$ jsou \textit{strukturní neznámé}.

Účelová funkce se buď maximalizuje

\begin{equation}
    \max z = c_1x_1 + c_2x_2 + \ldots + c_nx_n ,
\end{equation}

nebo minimalizuje

\begin{equation}
    \min z = c_1x_1 + c_2x_2 + \ldots + c_nx_n .
\end{equation}

\subsubsection{Omezující podmínky}

jsou lineární rovnice nebo nerovnice ve tvaru

\begin{equation}
    \begin{gathered}
        a_{11}x_1 + a_{12}x_2 + \ldots + a_{1n}x_n \ \lesseqgtr \ b_1, \\
        a_{21}x_1 + a_{22}x_2 + \ldots + a_{2n}x_n \ \lesseqgtr \ b_2, \\
        \vdots \\
        a_{m1}x_1 + a_{m2}x_2 + \ldots + a_{mn}x_n \ \lesseqgtr \ b_m,
    \end{gathered}
\end{equation}

kde na místě označeném $\lesseqgtr$ se může vyskytnout symbol $\leq$, $\geq$ nebo $=$\footnote{Tato formulace je doslovným přepisem z \textit{Operační výzkum} J. Demel \cite{demel} str. 9.}.

Prvky $a_{ij}$ jsou konstanty, které nazýváme \textit{strukturní koeficienty} nebo \textit{koeficienty omezení},
$b_1, b_2, \ldots, b_m$
jsou konstanty (tzv. \textit{požadavková čísla})
jsou konstanty, které nazýváme \textit{strukturní koeficienty} nebo \textit{koeficienty omezení}, $b_i$ jsou konstanty (tzv. \textit{požadavková čísla})
a $x_1, x_2, \ldots x_n$ jsou \textit{strukturní neznámé}.

Zároveň omezující podmínky vymezují pro každou proměnnou $x_1, x_2, \ldots x_n$ množinu hodnot, kterých může nabývat. 
Nejčastěji se jedná o podmínky tvaru $x_i \geq 0$ (nezápornost).
Jinými případy mohou být například podmínky tvaru $x_i \leq 0$ (nekladnost) nebo $x_i$ může nabývat libovolné hodnoty („neomezeno“).

\subsection{Maticový zápis úlohy LP}

Celý problém lineárního programování můžeme pro přehlednost zapsat maticově.
Účelovou funkci vyjádříme jako
jako

\begin{equation}
    z = \bm{c}^T\bm{x} \rightarrow \max,
\end{equation}

nebo

\begin{equation}
    z = \bm{c}^T\bm{x} \rightarrow \min,
\end{equation}

kde $\bm{c}$ je sloupcový vektor cenových koeficientů a $\bm{x}$ je sloupcový vektor \textit{strukturních neznámých}.

\textit{Strukturní koeficienty} můžeme vyjádřit jako matici

\begin{equation}
    \bm{A} = 
    \begin{pmatrix}
        a_{11} & a_{12} & \ldots & a_{1n} \\
        a_{21} & a_{22} & \ldots & a_{2n} \\
        \vdots & \vdots & \ddots & \vdots \\
        a_{m1} & a_{m2} & \ldots & a_{mn}
    \end{pmatrix}
\end{equation}

a za předpokladu, že jsou všechny omezující podmínky stejného typu (tzn. $\leq$, $\geq$ nebo $=$), můžeme je vyjádřit jako 

\begin{equation}
    \begin{split}
        \bm{A}\bm{x} &\leq \bm{b} \quad \text{nebo} \quad 
        \bm{A}\bm{x} \geq \bm{b} \quad \text{nebo} \quad 
        \bm{A}\bm{x} = \bm{b},
    \end{split}
    \end{equation}

kde $\bm{b}$ je sloupcový vektor \textit{požadavkových čísel}.

\subsection{Typy úloh lineárního programování}

V následující časti jsou popsány některé typické úlohy lineárního programování \footnote{Podle \textit{Matematika pro ekonomy} R. Stolín \cite{matematika_pro_ekonomy}}.

\subsubsection{Úlohy výrobního plánování}

\subsubsection{Směšovací úlohy}

\subsubsection{Úlohy o dělení materiálu}