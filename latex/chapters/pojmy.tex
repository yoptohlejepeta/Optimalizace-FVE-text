\section{Přehled ekonomických pojmů}

\subsection{?Základní pojmy?}

\subsection{Ukazatele výnosnosti investice}


\subsubsection{Výnosnost investice}
($ROI$ -- Return of Investment)
% https://www.moneta.cz/slovnik-pojmu/detail/roi
vyjadřuje zisk nebo ztrátu z investice v procentech.

\begin{equation}
    ROI = \frac{(P_1 + P_2 + \ldots + P_t) - K}{K} \cdot 100,
\end{equation}

kde
\begin{itemize}[label={}]
    \item $t$ -- počet let,
    \item $P_1, P_2, \ldots, P_t$ -- peněžní příjmy z investice v jednotlivých letech,
    \item $K$ -- kapitálový výdaj,
    \item $ROI$ -- návratnost investice.
\end{itemize}

\subsubsection{Diskontované cash-flow}
($DCF$ -- Discounted Cash Flow)
vyjadřuje současnou hodnotu budoucích peněžních toků.

\begin{equation}
    DCF = \frac{P_1}{(1+i)} + \frac{P_2}{(1+i)^2} + \ldots + \frac{P_t}{(1+i)^t},
\end{equation}

kde
\begin{itemize}[label={}]
    \item $t$ -- počet let,
    \item $P_1, P_2, \ldots, P_t$ -- peněžní příjmy z investice v jednotlivých letech,
    \item $i$ -- úroková míra (diskontní sazba),
    \item $DCF$ -- diskontované cash-flow.
\end{itemize}

\subsubsection*{Čistá současná hodnota}
($NPV$ -- Net Present Value)
vyjadřuje současnou hodnotu budoucích peněžních toků po odečtení kapitálového výdaje.

\begin{equation}    
NPV = \frac{P_1}{(1+i)} + \frac{P_2}{(1+i)^2} + \ldots + \frac{P_t}{(1+i)^t} - K,
\end{equation}

kde
\begin{itemize}[label={}]
    \item $t$ -- počet let,
    \item $P_1, P_2, \ldots, P_t$ -- peněžní příjmy z investice v jednotlivých letech,
    \item $K$ -- kapitálový výdaj,
    \item $i$ -- úroková míra (diskontní sazba),
    \item $NPV$ -- čistá současná hodnota.
\end{itemize}

\subsubsection*{Vnitřní výnosové procento}
($IRR$ -- Internal Rate of Return)
je úroková míra, při níž se současná hodnota peněžních příjmů z investice rovná kapitálovým výdajům. Investice se považuje za výhodnou, když $IRR$ představuje vyšší úrok, než je požadovaná minimální výnosnost investice.

\begin{equation}
    \frac{P_1}{(1+IRR)} + \frac{P_2}{(1+IRR)^2} + \ldots + \frac{P_t}{(1+IRR)^t} = K,
\end{equation}

kde
\begin{itemize}[label={}]
    \item $t$ -- počet let,
    \item $P_1, P_2, \ldots, P_t$ -- peněžní příjmy z investice v jednotlivých letech,
    \item $K$ -- kapitálový výdaj,
    \item $IRR$ -- vnitřní výnosové procento.
\end{itemize}
