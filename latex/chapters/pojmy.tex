Tato kapitola je rozdělena do dvou částí. První část se zabývá pojmy a ukazateli z oblasti investic.
V druhé části jsou popsány metody a základní modely lineárního programování.

\section{Investice}

% TODO: citace

Investicí se označuje takové vynaložení finančních prostředků, které se v budoucnu nějakou mírou zhodnotí.
Investice se dělí na reálné a finanční. Mezi reálné patří například investice do nemovitostí, drahých kovů nebo uměleckých děl, tedy do hmotného majetku.
Finanční investice jsou investice do finančních aktiv, jako jsou akcie nebo dluhopisy.
Jako základní pomůcka k posouzení investic slouží \textit{investiční trojúhelník}, který zahrnuje tři základní parametry investice: rizikovost, výnosnost a likviditu.
Obecně platí, že s rostoucím výnosem roste rizikovost a klesá likvidita a naopak.

\subsection{Investiční možnosti}

Záměrem je porovnat investici do fotovoltaické elektrárny s vybranými finančními investicemi.


\subsubsection{Spořící účty}
mají zpravidla vyšší úrokovou sazbu než běžné účty.
Výhodou je vysoká likvidita.
V současnosti se úroková sazba pohybuje okolo 4.5\% p.a.

\subsubsection{Dluhopisy} jsou cenné papíry,
které držiteli garantuje pravidelný výnos a jeho následný zpětný odkup.
Známým druhem dluhopisů jsou státní dluhopisy, které Ministerstvo financí prodává několika největším bankám a obchodníkům.
Ti je pak prodávají dalším zájemcům.


\subsubsection{Akcie} je cenný papír, který představuje podíl na kapitálu akciové společnosti.
Majitel akcií má právo na tzv. \textit{dividenda}, tedy podíl na zisku společnosti.


\subsection{Ukazatele výnosnosti investice}


\subsubsection{Výnosnost investice}
($ROI$ -- Return of Investment)
% https://www.moneta.cz/slovnik-pojmu/detail/roi
vyjadřuje zisk nebo ztrátu z investice v procentech.

\begin{equation}
    ROI = \frac{(P_1 + P_2 + \ldots + P_t) - K}{K} \cdot 100,
\end{equation}

kde
\begin{itemize}[label={}]
    \item $t$ -- počet let,
    \item $P_1, P_2, \ldots, P_t$ -- peněžní příjmy z investice v jednotlivých letech,
    \item $K$ -- kapitálový výdaj,
    \item $ROI$ -- návratnost investice.
\end{itemize}

\subsubsection{Diskontované cash-flow}
($DCF$ -- Discounted Cash Flow)
vyjadřuje současnou hodnotu budoucích peněžních toků.

\begin{equation}
    DCF = \frac{P_1}{(1+i)} + \frac{P_2}{(1+i)^2} + \ldots + \frac{P_t}{(1+i)^t},
\end{equation}

kde
\begin{itemize}[label={}]
    \item $t$ -- počet let,
    \item $P_1, P_2, \ldots, P_t$ -- peněžní příjmy z investice v jednotlivých letech,
    \item $i$ -- úroková míra (diskontní sazba),
    \item $DCF$ -- diskontované cash-flow.
\end{itemize}

\subsubsection*{Čistá současná hodnota}
($NPV$ -- Net Present Value)
vyjadřuje současnou hodnotu budoucích peněžních toků po odečtení kapitálového výdaje.

\begin{equation}    
NPV = \frac{P_1}{(1+i)} + \frac{P_2}{(1+i)^2} + \ldots + \frac{P_t}{(1+i)^t} - K,
\end{equation}

kde
\begin{itemize}[label={}]
    \item $t$ -- počet let,
    \item $P_1, P_2, \ldots, P_t$ -- peněžní příjmy z investice v jednotlivých letech,
    \item $K$ -- kapitálový výdaj,
    \item $i$ -- úroková míra (diskontní sazba),
    \item $NPV$ -- čistá současná hodnota.
\end{itemize}

\subsubsection*{Vnitřní výnosové procento}
($IRR$ -- Internal Rate of Return)
je úroková míra, při níž se současná hodnota peněžních příjmů z investice rovná kapitálovým výdajům. Investice se považuje za výhodnou, když $IRR$ představuje vyšší úrok, než je požadovaná minimální výnosnost investice.

\begin{equation}
    \frac{P_1}{(1+IRR)} + \frac{P_2}{(1+IRR)^2} + \ldots + \frac{P_t}{(1+IRR)^t} = K,
\end{equation}

kde
\begin{itemize}[label={}]
    \item $t$ -- počet let,
    \item $P_1, P_2, \ldots, P_t$ -- peněžní příjmy z investice v jednotlivých letech,
    \item $K$ -- kapitálový výdaj,
    \item $IRR$ -- vnitřní výnosové procento.
\end{itemize}
